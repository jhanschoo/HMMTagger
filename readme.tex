% For more powerful document formatting
\documentclass[12pt,article,oneside,onecolumn,final,a4paper]{memoir}

% For more powerful command and environment definition
\usepackage{xparse}

% titling content
\NewDocumentCommand{\modname}{}{CS4248}
\NewDocumentCommand{\assignname}{}{Assignment 2 Report}
\NewDocumentCommand{\datename}{}{\today}

% For color customization and drawing
\usepackage{xcolor}
\usepackage{pagecolor}
\usepackage{framed}
\usepackage{pstricks}

\definecolor{tnbg}{HTML}{FFFFFF}
\definecolor{tncl}{HTML}{EFEFEF}
\definecolor{tnse}{HTML}{D6D6D6}
\definecolor{tnfg}{HTML}{4D4D4C}
\definecolor{tnco}{HTML}{8E908C}
\definecolor{tnre}{HTML}{C82829}
\definecolor{tnor}{HTML}{F5871F}
\definecolor{tnye}{HTML}{EAB700}
\definecolor{tngr}{HTML}{718C00}
\definecolor{tnaq}{HTML}{3E999F}
\definecolor{tnbl}{HTML}{4271AE}
\definecolor{tnpu}{HTML}{8959A8}

% For math support
\usepackage{amsmath,amssymb}
\usepackage[amsmath,framed,hyperref,thmmarks]{ntheorem}

% Fix \left \right
\usepackage{mleftright}

% For OpenType typographic support
\usepackage{fontspec}
\usepackage{realscripts}
\usepackage{unicode-math}
\usepackage[final]{microtype}

% Using Chivo for display text, Cochineal for body, and Pagella for
% math c.f. fontspec and unicode-math
\setmainfont{texgyretermes}[
Extension = .otf,
UprightFont = *-regular,
ItalicFont = *-italic,
BoldFont = *-bold,
BoldItalicFont = *-bolditalic,
Ligatures = TeX
]
\setmathfont{texgyretermes-math.otf}[Scale=MatchLowercase]
\setmathfontface\mathit{texgyretermes-italic.otf}[Scale=MatchLowercase]
\setmathfontface\mathbf{texgyretermes-bold.otf}[Scale=MatchLowercase]
\newfontfamily\chivo{Chivo}[
Extension = .otf,
UprightFont = *-Regular,
ItalicFont = *-Italic,
BoldFont = *-Bold,
BoldItalicFont = *-BoldItalic,
Ligatures = TeX
]
\newfontfamily\chivobold{Chivo}[
Extension = .otf,
UprightFont = *-Bold,
ItalicFont = *-BoldItalic,
BoldFont = *-Black,
BoldItalicFont = *-BlackItalic,
Ligatures = TeX
]
\newfontfamily\chivolight{Chivo}[
Extension = .otf,
UprightFont = *-Light,
ItalicFont = *-LightItalic,
BoldFont = *-Regular,
BoldItalicFont = *-Italic,
Ligatures = TeX
]

% For formatting itemize and enumerate environments
\usepackage{enumitem}

% For better tables
\usepackage{longtable}
\usepackage{tabu}

% PDF features
\usepackage[final,unicode,
colorlinks,
linkcolor=tnre,
anchorcolor=tnfg,
citecolor=tngr,
filecolor=tnaq,
menucolor=tnre,
urlcolor=tnpu
]{hyperref}
\usepackage{footnotebackref}

\usepackage{cleveref}

% To mark sections as TODO
\usepackage[draft]{fixme}

% To show labels
%\usepackage[inline]{showlabels}

%\pagecolor{tnbg}
%\color{tnfg}

% define augmented matrices
\makeatletter
\renewcommand*\env@matrix[1][*\c@MaxMatrixCols c]{%
  \hskip -\arraycolsep
  \let\@ifnextchar\new@ifnextchar
  \array{#1}}
\makeatother

% microtype compatibility workaround
% c.f. https://tex.stackexchange.com/questions/373594/microtype-producing-dozens-of-unknown-slot-number-warnings
\makeatletter
\def\MT@is@composite#1#2\relax{%
  \ifx\\#2\\\else
    \expandafter\def\expandafter\MT@char\expandafter{\csname\expandafter
                    \string\csname\MT@encoding\endcsname
                    \MT@detokenize@n{#1}-\MT@detokenize@n{#2}\endcsname}%
    % 3 lines added:
    \ifx\UnicodeEncodingName\@undefined\else
      \expandafter\expandafter\expandafter\MT@is@uni@comp\MT@char\iffontchar\else\fi\relax
    \fi
    \expandafter\expandafter\expandafter\MT@is@letter\MT@char\relax\relax
    \ifnum\MT@char@ < \z@
      \ifMT@xunicode
        \edef\MT@char{\MT@exp@two@c\MT@strip@prefix\meaning\MT@char>\relax}%
          \expandafter\MT@exp@two@c\expandafter\MT@is@charx\expandafter
            \MT@char\MT@charxstring\relax\relax\relax\relax\relax
      \fi
    \fi
  \fi
}
% new:
\def\MT@is@uni@comp#1\iffontchar#2\else#3\fi\relax{%
  \ifx\\#2\\\else\edef\MT@char{\iffontchar#2\fi}\fi
}
\makeatother
\makeatletter
\newtheoremstyle{myplain}%
{\item[\hskip\labelsep \theorem@headerfont ##1\ ##2\theorem@separator]}%
{\item[\hskip\labelsep \theorem@headerfont ##1\
  ##2]{\theorem@headerfont\ (##3)\theorem@separator\hskip\labelsep}}%
\newtheoremstyle{nonumbermyplain}%
{\item[\hskip\labelsep \theorem@headerfont ##1\theorem@separator]}%
{\item[\hskip\labelsep \theorem@headerfont ##1]{\theorem@headerfont\ (##3)\theorem@separator\hskip\labelsep}}%
\newtheoremstyle{nonumberproof}%
{\item[\hskip\labelsep \theorem@headerfont ##1\theorem@separator]}%
{\item[\hskip\labelsep \theorem@headerfont ##3\theorem@separator]}
\newtheoremstyle{nonumbersolution}%
{\item[\hskip\labelsep \theorem@headerfont ##1\theorem@separator]}%
{\item[\hskip\labelsep \theorem@headerfont\color{tngr} ##3\theorem@separator]
  \makeatother}

\theoremstyle{myplain}
\theoremheaderfont{\normalfont\chivo\bfseries}
\theoremsymbol{\ensuremath{_\Box}}
\theoremseparator{.}
\newtheorem{thm}{Theorem}[chapter]
\newtheorem{cor}[thm]{Corollary}
\newtheorem{lem}[thm]{Lemma}

\theoremstyle{nonumbermyplain}
%\newtheorem{prop}{Proposition}

\theoremstyle{myplain}
\theorembodyfont{\upshape}
\newtheorem{xca}[thm]{Exercise}
\newtheorem{exmp}[thm]{Example}
\newtheorem{defn}[thm]{Definition}

\theoremstyle{nonumbermyplain}
\newtheorem{rem}[thm]{Remark}

\theoremheaderfont{\normalfont\chivolight%\itshape%
%\scshape
}
\theorembodyfont{\upshape}
\theoremsymbol{\ensuremath{_\blacksquare}}
\theoremstyle{nonumberproof}
\newframedtheorem{prf}{Proof}
\theoremstyle{nonumbersolution}
\newframedtheorem{soln}{Solution}

% Customize equation numbering, cf. amsmath
\renewcommand{\theequation}{\thechapter--\arabic{equation}}
\numberwithin{equation}{chapter}

% customization of section headers cf. memoir

\counterwithout{section}{chapter}

\setsecnumdepth{subsection}

\RenewDocumentCommand\beforepartskip{}{\null\vskip 50pt}
\RenewDocumentCommand\afterpartskip{}{\vskip 40pt}
\RenewDocumentCommand\partnamefont{}{\normalfont\chivolight\Huge}
\RenewDocumentCommand\partnumfont{}{\normalfont\chivolight\Huge}
\RenewDocumentCommand\parttitlefont{}{\normalfont\chivolight\bfseries\HUGE}
\nopartblankpage{}

\chapterstyle{default}
\RenewDocumentCommand\chapterheadstart{}{\vspace*{30pt}}
\RenewDocumentCommand\afterchapternum{}{\par\nobreak\vskip 5pt}
\RenewDocumentCommand\afterchaptertitle{}{\par\nobreak\vskip 10pt}
\RenewDocumentCommand\chapnamefont{}{\normalfont\chivolight\bfseries\huge}
\RenewDocumentCommand\chapnumfont{}{\normalfont\chivolight\bfseries\huge}
\RenewDocumentCommand\chaptitlefont{}{\normalfont\chivobold\Huge\bfseries}

\setsecheadstyle{\normalfont\chivobold\Large}
\setsubsecheadstyle{\normalfont\chivo\bfseries\large}
\setsubsubsecheadstyle{\normalfont\chivo\large}
\setparaheadstyle{\normalfont\chivo\bfseries\large}
\setsubparaheadstyle{\normalfont\chivo\large}


% customization of page styles cf. memoir

\makeheadrule{plain}{\textwidth}{\normalrulethickness}

\makeoddhead{plain}{}{}{}
\makeevenhead{plain}{}{}{}

\makeoddfoot{plain}{}{\chivolight—\,\thepage\,—}{}
\makeevenfoot{plain}{}{\chivolight—\,\thepage\,—}{}
% customization of itemize and enumerate environments cf. enumitem
\setlist{
  noitemsep,
  nosep,
  font=\normalfont\chivo\bfseries
}

\setlist[itemize]{
  label=\textbullet{}
}

\newlist{alphenum}{enumerate}{4}
\setlist*[alphenum]{
  label= (\alph*)
}

\newlist{romanenum}{enumerate}{4}
\setlist*[romanenum]{
  label= (\roman*)
}

% customization of tabu environments
\global\tabulinesep=4pt

% customize FiXme labels
\fxsetup{
layout={inline},
}

% redefine FiXme labels to have brackets
\makeatletter
\renewcommand*\FXLayoutInline[3]{%
  {\@fxuseface{inline} (\fxnotename{#1}: \textnormal{#2})}%
}
\makeatother

% Cleveref counter names
\crefname{thm}{thm.}{thms.}
\crefname{lem}{lem.}{lems.}
\crefname{prop}{prop.}{props.}
\crefname{cor}{cor.}{cors.}
\crefname{defn}{defn.}{defns.}
\crefname{rem}{rem.}{rems.}
\crefname{exmp}{exmp.}{exmps.}
\crefname{xca}{exerc.}{exercs.}
\crefname{step}{step}{steps}
\crefname{add}{add.}{adds.}
\crefname{prf}{proof}{proofs}
\Crefname{thm}{Theorem}{Theorems}
\Crefname{lem}{Lemma}{Lemmata}
\Crefname{prop}{Proposition}{Propositions}
\Crefname{cor}{Corollary}{Corollaries}
\Crefname{defn}{Definition}{Definitions}
\Crefname{rem}{Remark}{Remarks}
\Crefname{exmp}{Example}{Examples}
\Crefname{xca}{Exercise}{Exercises}
\Crefname{step}{Step}{Steps}
\Crefname{add}{Addendum}{Addenda}
\Crefname{prf}{Proof}{Proofs}

% SHORTHAND

% text shorthand

\NewDocumentCommand{\nt}{m}{\textbf{#1}}

\NewDocumentCommand{\todo}{O{} g}{\IfNoValueTF{#2}%
  {\fxnote[#1]{\textbf{TODO}}}%
  {\fxnote[#1]{\textbf{TODO:} #2}}
}
\NewDocumentCommand{\wontfix}{O{} g}{\IfNoValueTF{#2}%
  {\fxnote[#1]{\textbf{WON'T FIX}}}%
  {\fxnote[#1]{\textbf{WON'T FIX:} #2}}
}

% math shorthand

% delimiters
\NewDocumentCommand{\abs}{m}{\mleft\lvert{#1}\mright\rvert}
\NewDocumentCommand{\norm}{m}{\mleft\lVert{#1}\mright\rVert}
\NewDocumentCommand{\fl}{m}{\mleft\lfloor{#1}\mright\rfloor}
\NewDocumentCommand{\ce}{m}{\mleft\lceil{#1}\mright\rceil}
\NewDocumentCommand{\p}{m}{\mleft({#1}\mright)}
\NewDocumentCommand{\ad}{m}{\mleft\langle{#1}\mright\rangle}
\NewDocumentCommand{\B}{m}{\mleft\{{#1}\mright\}}
\NewDocumentCommand{\s}{m}{\mleft[{#1}\mright]}

% math operators
\DeclareMathOperator{\D}{d\!}
\DeclareMathOperator{\rank}{rank}
\DeclareMathOperator{\nullity}{nullity}
\DeclareMathOperator{\rowrank}{row\ rank}
\DeclareMathOperator{\columnrank}{column\ rank}
\DeclareMathOperator{\tr}{tr}
\DeclareMathOperator{\adj}{adj}
\DeclareMathOperator{\sgn}{sgn}

\NewDocumentCommand{\bbZ}{}{\symbb{Z}}

% decoration shorthand
\NewDocumentCommand{\conj}{m}{\overline{#1}}
\NewDocumentCommand{\clos}{m}{\overline{#1}}

% vector shorthand (w as a commonly used vector variable character)
\NewDocumentCommand{\w}{m}{\symbfup{#1}}

% Use \m to typeset math variables that take up multiple symbols.
\NewDocumentCommand{\q}{m}{\mathit{#1}}

% divides |
\NewDocumentCommand{\mdi}{}{\mid}
\NewDocumentCommand{\nmdi}{}{\nmid}

% Catenation
\NewDocumentCommand{\cat}{}{\cdot}

% zero word
\NewDocumentCommand{\ew}{}{\epsilon}

% index sequence
\NewDocumentCommand{\sei}{s O{1} m}%
{
  \IfBooleanTF #1%
  {\({#2},\dotsc,{#3}\)}%
  {{#2},\dotsc,{#3}}%
}

% extensional sequence
\NewDocumentCommand{\se}{s m O{1} m}%
{%
  \IfBooleanTF #1%
  {\({#2}_{#3},\dotsc,{#2}_{#4}\)}%
  {{#2}_{#3},\dotsc,{#2}_{#4}}%
}

\NewDocumentCommand{\makeassignmenttitle}{}{
  \begin{flushleft}
    Johannes Choo\\
    \datename{}
  \end{flushleft}
  \rule{\textwidth}{\normalrulethickness}
}

\begin{document}
\makeassignmenttitle{}

\setcounter{chapter}{1}

\setcounter{thm}{4}

\section{Extending Kneser-Ney to Closure in Unseen \(N\)-grams}

We build in python2 a HMM tagger employing Interpolated Kneser-Ney
smoothing as described by Jurafsky and Martin (August 2017 Draft). The
description given by Jurafsky and Martin is not complete, and notably
not closed under unobserved tokens and contexts. We extend it to
closure over unobserved \(N\)-grams with the following probability
measure defined over words \(w\) with either no context, or a context
of arbitrary positive length \(c'\w{c}\). We have
\begin{gather*}
  P_{KN}\p{w\mid c'\w{c}}=
  \begin{cases}
    \frac{\max{\p{c\p{c'\w{c}w}-d,0}}}{S\p{c'\w{c}}}
    +\lambda\p{c'\w{c}}P_{KN}\p{w\mid\w{c}} &
    \text{if \(S\p{c'\w{c}}>0\),}\\
    P_{KN}\p{w\mid\w{c}} &
    \text{if \(S\p{c'\w{c}}=0\),}\\
  \end{cases} \\
  P_{KN}\p{w}=P_{KN}\p{w\mid\epsilon}=
  \begin{cases}
    \frac{\max\p{c\p{w}-d,0}}{S\p{\epsilon}}+\lambda\p{\epsilon}\frac{1}{\abs{T}+1}
    &\text{if \(S\p{\epsilon}>0\),} \\
    \frac{1}{\abs{T}+1}&\text{if \(S\p{\epsilon}=0\),}
  \end{cases}
\end{gather*}
where we wish to obtain the probability of a token \(w\) after
observing context \(c'\w{c}\). \(T\) is the (unigram) vocabulary. For
bigrams, we let \(\w{c}=\epsilon\), the empty tuple of words, in order that
the context is just a single word \(c'\). \(d\) is a parameter with
\(0\leq d\leq 1\) of the absolute discount to the \(N\)-gram counts; it is
to be learned by experimentation. \(S\p{\w{c}}\) is a normalization
term given by
\begin{equation*}
  S\p{\w{c}}=\sum_{w\in T}c\p{\w{c}w},
\end{equation*}
including for \(\w{c}=\epsilon\). \(\lambda\) redistributes the discounted probability mass
uniformly over all observed tuples the size of \(c'\w{c}w\), given by
\begin{equation*}
  \lambda\p{\w{c}}=\frac{d}{S\p{\w{c}}}\abs{\B{w\in T:c\p{\w{c}w}>0}},
\end{equation*}
including for \(\w{c}=\epsilon\). The count \(c\) is defined by
\begin{equation*}
  c\p{\w{u}}=
  \begin{cases}
    c_t\p{\w{u}}&\text{if \(\abs{\w{u}}=N-1\),} \\
    \abs{\B{v:c_t\p{v\w{u}}>0}}&\text{if \(\abs{\w{u}}<N-1\).}
  \end{cases}
\end{equation*}
including for \(\w{u}=\epsilon\) where \(c_t\) are the raw observations from
the training set and \(N\) is the \(N\) of our \(N\)-gram model
(i.e. the number of atoms).

Our implementation uses a straightforward approach where we compute
\(c_t\), \(c\), \(T\), \(S\) during training, and compute \(P_{KN}\)
at the time of query. We slide a window of maximum size \(N\) across
the training corpus, and for each final token \(w\), we record the
\(N\) tuples that it generates, from the \(N\)-gram to the unigram. In
our implementation, we perform smoothing separately for the (hidden)
Markov chain and for the tag-token emission probabilities. We use
\(N=2\) for a bigram HMM model.

\section{Implementation of the Viterbi Algorithm}

We implement the Viterbi algorithm in a straightforward translation of
the description provided by Jurafsky and Martin, with a couple of
modifications. The computation of the tag transition and the token
emission probabilities are done in probability masses themselves
(since we have to add probabilities from different models in KN
smoothing) but once computed, they are converted and accumulated as
logprobs.

We make a second modification: we only consider candidate tags for
which there has been an observed transition from the context tag. This
slightly reduces the accuracy when the correct tag involves an
unobserved transition from the current context, but greatly speeds up
the algorithm and its asymptotic performance if we reasonably assume
that the (observed) transition matrix from context to tag is sparse.

\section{Training and Running the Tagger}

To train the tagger, run:
\begin{verbatim}
python2 build_tagger.py sents.train sents.devt model_file
\end{verbatim}

To run the tagger on a test file, run:
\begin{verbatim}
python2 run_tagger.py sents.test model_file sents.out
\end{verbatim}

In both cases, the \texttt{HMMTagger.py} file should be present in the
same directory as the script file (\texttt{build\_tagger.py} or
\texttt{run\_tagger.py}).

\section{References}

\begin{description}
\item Daniel Jurafsky and James H. Martin.
  2017. \textit{Speech and Language Processing (3rd Edition,
    August Draft).} [online] Available at
  https://web.stanford.edu/\textasciitilde jurafsky/slp3/ [Accessed 28 Sep.\
  2017].
\end{description}

\end{document}

%%% Local Variables:
%%% mode: latex
%%% TeX-engine: luatex
%%% TeX-master: t
%%% End:
